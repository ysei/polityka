\documentclass[10pt]{article}

% Autorzy: (proszę się dopisywać)
%	Piotr Zalewa <piotr@zalewa.info>

\usepackage[utf8x]{inputenc}
\usepackage{polski, fullpage}

\title{"By żyło się lepiej", czyli trzy lata psucia państwa\\wyjątki}
\author{Tekst oryginału --- Tomasz Dalecki --- Sekretarz Generalny UPR\\Redakcja wyjątków, wybór źródeł --- praca zbiorowa}

\begin{document}
\maketitle
\thispagestyle{empty}

\noindent Pełen tekst ukazał się na stronach internetowych Unii Polityki Realnej.\footnote{http://uniapolitykirealnej.pl/wiadomoci/253-by-yo-si-lepiej-czyli-trzy-lata-psucia-pastwa}

Na początek kilka słów wyjaśnienia – nie jestem, nie byłem i nie będę zwolennikiem PiS. Szczerze nie znoszę tej partii i życzę jej jak najgorzej. Broń Boże nie wskazuję na PiS jako na właściwą alternatywę dla nieudolnych rządów PO.

Wyjątek z Deklaracji Ideowej PO:\footnote{http://staremiasto.krakow.platforma.org/index.php/deklaracja-ideowa-po.html}

\noindent {\em Niestety, po 12 latach niepodległości, Polska znowu przestała doganiać uciekający w szybkim rozwoju świat. Polskę ogarnia fala stagnacji i niewiary w przyszłość. Główną przyczyną jest paraliżowanie rozwijającej się przedsiębiorczości i obywatelskiej inicjatywy przez biurokrację, złe prawo i grupowe interesy związków zawodowych.}

\noindent Te słowa są dziś bardziej aktualne niż kiedykolwiek w ciągu ostatniego dwudziestolecia. {\bf Polityka PO jest zaprzeczeniem liberalizmu. Rządy Donalda Tuska są zaprzeczeniem idei dobrego rządzenia.} W rzeczywistości jest to marsz w kierunku katastrofy finansów państwa, społecznego rozkładu i faszyzacji życia publicznego.

W kampanii w 2005 roku PO szermowała hasłem „3x15”, postulując wprowadzenie liniowej stawki PIT w wys. 15\%, obniżkę CIT do takiej właśnie wysokości i ujednolicenie stawek podatku VAT na takim właśnie poziomie. W swoim expose premier Donald Tusk mówił m.in.: {\em Naczelną zasadą polityki finansowej mojego rządu będzie stopniowe obniżanie podatków i innych danin publicznych. [...] Będziemy prowadzili tę politykę rozważnie, to musi być rozważny marsz. Ale chcę, aby to był marsz zawsze w jednym kierunku, zawsze w kierunku niższych podatków i zawsze w kierunku rezygnacji z nadmiernych, często zbędnych danin publicznych, jakie obywatel płaci na rzecz administracji.}

Dzień wolności podatkowej\footnote{http://pl.wikipedia.org/wiki/Dzie\%C5\%84\_Wolno\%C5\%9Bci\_Podatkowej} za czasów rządów PO systematycznie przesuwa się do przodu. W latach 2007 i 2008, w wyniku nieznacznej obniżki obciążeń, które zawdzięczamy koalicji pod wodzą PiS, dzień wolności podatkowej cofnął się o kilka dni. Od roku 2008 to radosne święto wypada coraz później, w tym roku roku opóźnienie w stosunku do roku poprzedniego wynosi aż 9 dni!

{\bf Sposób wprowadzenia podwyżki:} Najpierw wypuszczony został balon próbny w postaci zapowiedzi podwyżki stawki VAT aż o 3 punkty procentowe\footnote{http://www.forbes.pl/artykuly/sekcje/wydarzenia/podatnicy--zacznijcie-sie-bac!,5568,1}. Do ogłoszenia tych propozycji wybrany został minister Michał Boni. Stwierdził on, że trzeba wynagrodzić budżetowi ,,ubytki", jakie poniósł w wyniku obniżki PIT i składki rentowej za rządów PiS. Następnie Propozycję Boniego skrytykował Grzegorz Schetyna\footnote{http://wiadomosci.gazeta.pl/Wiadomosci/1,81048,8196506,Schetyna\_\_we\_wladzach\_PO\_jednolite\_zdanie\_ws\_\_podwyzki.html}, by później PO mogła poinformować o ,,kompromisowej" propozycji podwyżki o 1 punkt procentowy (to zresztą nie jest do końca prawda, stawki 21 i 7 procent zostaną podniesione o jeden punkt, ale stawka 3\% o 2 punkty)\footnote{http://www.tvn24.pl/12690,8530,1667275,,,rzad-podniesie-vat-o-1-proc--tusk-wariant-srodka-drogi,raport\_wiadomosc.html}. Tym razem do ogłoszenia tych ,,radosnych wieści" wybrany został nowy szef klubu poselskiego PO, pan poseł Tomczykiewicz. Całość sprawia nieco orwellowskie wrażenie, jakoby to dobry ,,Wielki Brat” Tusk, wbrew zakusom niektórych, w istocie postanowił obniżyć podatki.

Oczywiście, podwyżka VAT jest tylko ,,czasowa". Jak wyglądają takie ,,czasowe podwyżki” najlepiej chyba ilustruje historia ,,belkowego” istnieje już ładnych kilka lat i nie zanosi się na jego likwidację. PO w swoim programie wyborczym zapowiadała właśnie likwidację belkowego\footnote{http://www.platforma.org/pl/program/ str 29}. W sejmie obecnej kadencji był nawet projekt odpowiedniej ustawy w tej sprawie autorstwa... Prawa i Sprawiedliwości\footnote{XXX: dodaj źródło}. Projekt nie wszedł nawet pod obrady, został zablokowany przez marszałka sejmu Bronisława Komorowskiego\footnote{XXX: dodaj źródło}!

Niektóre obciążenia podnosi się po cichu. W przyszłym roku rząd zafunduje nam kolejne ograniczenia w odliczaniu VAT od paliw. Rezygnacja z ulgi podatkowej od biokomponentów dodawanych (przymusowo!) do paliw płynnych spowoduje podwyżkę cen benzyny o 5-7 gr. na litrze (a po ovatowaniu nową, wyższą stawką przełoży się to na ok. 9-11 gr. wzrostu).

Istnieją też daniny, które nie zasilają budżetu. Do takich należy tzw. opłata reprograficzna, którą obciążone są niektóre artykuły papiernicze i elektroniczne. Dochody z tej opłaty przekazywane są na rzecz ,,związków twórczych” i w założeniach mają stanowić rekompensatę części dochodów z praw autorskich, jakie twórcy tracą w wyniku kopiowania ich dzieł. Z inicjatywy Ministerstwa Kultury i „środowisk twórczych” szykowana jest nowelizacja ustawy\footnote{XXX: dodaj źródło}, która z jednej strony podwyższy opłaty (np. do 1,5\% na papier), z drugiej zaś rozszerzy zakres towarów objętych opłatą, m.in. o pendrive'y i cyfrowe aparaty fotograficzne.

Komisja „przyjazne państwo” kierowana przez posła Janusza Palikota zamiast ulżyć przedsiębiorcom wyprodukowała tylko kilka bubli prawnych, a żadnymi poważnymi problemami nawet się nie zajęła. Spektakularną wpadką komisji były słynna ustawa o strażakach\footnote{http://www.polityka.pl/kraj/279284,1,palacy-problem.read}...

Po wybuchu tzw. afery hazardowej „liberalny” rząd postanowił powalczyć z hazardem. W ramach walki z hazardem nie tylko ograniczono kolejny obszar naszej wolności, ale też zlikwidowano ok. 100 tys. miejsc pracy w branży hazardowej\footnote{XXX: dodaj źródło}. Bezpośrednie straty dla budżetu, tylko z tytułu niższych wpływów z podatku od gier, wynoszą w tym roku kilkaset milionów złotych\footnote{XXX: dodaj źródło}. Jak to się ma do Deklaracji Ideowej PO, w której czytamy m.in.

\noindent {\em Nie ma innej drogi do szybkiego rozwoju i dobrobytu jak powrót do idei wolności. Nie ma innej skutecznej polityki gospodarczej – jak polityka konkurencji, ochrony własności prywatnej i twardego rozprawienia się przez państwo z przyczynami paraliżu przedsiębiorczości. [...] Zdolność człowieka do inicjatywy i przedsiębiorczości stanowi źródło bogactwa społecznego, zaś wolny rynek jest najbardziej skutecznym narzędziem wykorzystywania zasobów i zaspokajania potrzeb.}

Rząd wpadł również na pomysł cenzurowania internetu\footnote{http://www.money.pl/gospodarka/ngospodarka/ebiznes/artykul/rzad;znow;bierze;sie;za;kontrole;sieci,146,0,578194.html}, w tym przede wszystkim stworzenia indeksu stron zakazanych, które dostawcy internetu mieliby obowiązek blokować. Już w założeniach na listę zakazanych stron trafiłaby nie tylko dziecięca pornografia, ale też strony siejące „mowę nienawiści”

Rzekomo liberalny rząd Donalda Tuska przygotował projekt nowej ustawy o CBA czyniącej z tej instytucji supersłużbę\footnote{http://wiadomosci.gazeta.pl/Wiadomosci/1,80708,8019794,Nowa\_ustawa\_o\_CBA\_\_\_Zamach\_na\_wolnosc\_obywatelska\_.html\\XXX: Doprecyzuj źródło. Najnowsza ustawa na stronach CBA BiP to 2006}. CBA zyskałaby wszelkie uprawnienia policyjne wobec wszystkich obywateli, włącznie z prawem do inwigilacji i do gromadzenia tzw. informacji wrażliwych.

Rząd wie również lepiej, jak obywatel ma dbać o swoje zdrowie. Palenie jest szkodliwe, więc troskliwy Donald Tusk postanowił uwolnić nas od tego nałogu. Prawo obywatela do dysponowania własnym zdrowiem rządu ani trochę nie obchodzi. Donald Tusk uznał też, że to „liberalna” władza decyduje o tym, co wolno, a co nie w restauracjach i barach, a właściciel nie ma w tej sprawie nic do gadania. 

Postulowana przez teoretyków i praktyków socjalistycznych likwidacja więzi rodzinnych znalazła swoje odbicie w tzw. ustawie o przemocy w rodzinie. Przepisy zakazujące rodzicom wychowywania dzieci, a w zamian umożliwiające urzędnikom rozbijanie rodzin koalicja rządząca uchwaliła z pomocą SLD, zaś p.o. prezydenta Bronisław Komorowski podpisał je niezwłocznie, nie czekając na rozstrzygnięcie wyborów prezydenckich. Ustawa zakazuje kar cielesnych wobec dzieci, pod demagogicznym hasłem przeciwdziałania znęcaniu się nad dziećmi (co było uregulowane prawnie już wcześniej). Ustawa ta odbiera rodzicom jakiekolwiek instrumenty wychowawcze --- bo każdy rodzaj perswazji może być zakwalifikowany jako przemoc fizyczna, psychiczna lub emocjonalna (w Szwecji podstawą do odebrania dzieci rodzicom był zakaz wyjścia na dyskotekę, będący przecież przejawem przemocy emocjonalnej). Szczytem hipokryzji wykazał się pan poseł Tomczykiewicz, który tak skomentował ustawę:
--- {\em Trochę żałuję, że nie zmieniliśmy tytułu tej ustawy. Wtedy byłaby do przyjęcia. Mówił też to obecny prezydent, kiedy ją podpisywał. Tytuł sugeruje, że w rodzinie jest przemoc, a jest ona zazwyczaj tam, gdzie tej rodziny nie ma.} --- Nie przeszkadza mu ustawowy maoizm a jedynie odbiór społeczny. 

Pani minister Katarzyna Hall, doprowadziła m.in. do odebrania dzieci rodzicom, by przymusowo umieścić je w przedszkolach\footnote{http://www.samorzad.pap.pl/palio/html.run?\_Instance=cms\_samorzad.pap.pl\&\_PageID=2\&s=depesza\\\&dz=szablon.depesza\&dep=39503\&data=\&\_CheckSum=489846056}. Jak przymus przedszkolny ma się do liberalizmu?

Po raz kolejny okazało się, że własna deklaracja ideowa jest dla PO nic nie znaczącym świstkiem papieru i nie stanowi żadnej przeszkody w rozwijaniu wynaturzonej inżynierii społecznej --- {\em Dlatego zadaniem Państwa jest roztropne wspieranie rodziny i tradycyjnych norm obyczajowych, służących jej trwałości i rozwojowi [...] Ideałowi obywatela – jako osoby wolnej i odpowiedzialnej za los swój i swojej rodziny.}

\end{document}
